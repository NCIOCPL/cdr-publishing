%% LaTeX Sample file for Patient Summaries.
%%                                   Created: 07/31/2001, Volker Englisch
%% ----------------------------------------------------------------------
%% The XML2LaTeX converter has to create a file with the data structure that
%%is presented here in order to display a properly formatted summary
%%

%% -- START -- Document Declarations and Definitions
\documentclass[12pt]{article}
%% Style file to format citations (exponented)
\usepackage{overcite}
%% Citations to be printed in bold
\renewcommand\citeform[1]{\bfseries #1}

% \usepackage{cite}
%% Style file to create literature reference for each chapter
\usepackage{chapterbib}

% Modify the default header "References" for Citation section
\renewcommand\refname{References:}

% Package and code to set correct double-quotes
% ---------------------------------------------
\usepackage{ifthen}
\newcounter{qC}
\newcommand{\tQ}{%
  \addtocounter{qC}{1}%
  \ifthenelse{\isodd{\value{qC}}}{``}{''}%
}

% Package Fancyhdr for Header/Footer/Reviewer Information
% -------------------------------------------------------
\usepackage{fancyhdr}
\pagestyle{fancy}

% Placing title, date, and board member in header
\fancyhead[C]{\bfseries \SummaryTitle \\}
\fancyhead[L]{\today}
\fancyhead[R]{\TheBoardMember}
\fancyfoot[C]{\thepage}
\renewcommand\headrulewidth{1pt}
\renewcommand\footrulewidth{1pt}

% Double-spacing
% \renewcommand\baselinestretch{1.5}
% Wider Margins
% \setlength{\oddsidemargin}{12pt}
% \setlength{\textwidth}{6in}

% Changing Font
% \renewcommand{\familydefault}{phv}

% \TheBoardMember --> XML = <BoardMember>
\newcommand{\TheBoardMember}{}
\newcommand{\BoardMember}[1]{\renewcommand{\TheBoardMember}{Reviewer:  #1}}

% \SummaryTitle --> XML = <SummaryTitle>
\newcommand{\SummaryTitle}{Ovarian Epithelial Cancer}

\setlength{\parskip}{1.2mm}
\setlength{\parindent}{4mm}

\renewcommand{\thesection}{\hspace{-1.0em}}

% Ensure that only sections are listed in TOC
\setcounter{tocdepth}{1}

%% -- END -- Document Declarations and Definitions

\begin{document}

% Tell fancyhdr package to modify the plain style (plain style is 
% default on a cover page), e.g. put header on first page as well.
% ----------------------------------------------------------------
\fancypagestyle{plain}{%
    % \fancyhf{}%
    \fancyhead[C]{\bfseries \SummaryTitle\\}
    \fancyhead[L]{\today}
    \fancyhead[R]{\TheBoardMember}
    \fancyfoot[C]{Font phv \\ \thepage}
    \renewcommand\headrulewidth{1pt}
    \renewcommand\footrulewidth{1pt}}

     \makeatletter \renewcommand\@biblabel[1]{#1.} \makeatother
 
\centerline{\bfseries \Large \SummaryTitle}
\BoardMember{V. Englisch}
\tableofcontents

\begin{cbunit}

% \section --> XML = <SummarySection>
% Paragraph --> XML = <Para>
\section{General Information}

\setcounter{qC}{0}
Note:  Separate PDQ summaries on Screening for Ovarian Cancer and Prevention of
Ovarian Cancer are also available.

\setcounter{qC}{0}
Note:  Some citations in the text of this section are followed by a level of
evidence.  The PDQ editorial boards use a formal ranking system to help the
reader judge the strength of evidence linked to the reported results of a
therapeutic strategy.  (Refer to the PDQ summary on Levels of Evidence for more
information.)

\setcounter{qC}{0}
% \cite + \bibitem --> <Citation>
This guy is working on the Footnote convertion.\cite{1,2}
The other guy is working on the CIPS project, too. \cite{2,3}
And at the end
we are using all three guys in one citation. \cite{2,3,1} 
At the very end
we need a couple couple more guys here. \cite{5,3,1,2,4} 


\setcounter{qC}{0}
Ovarian cancer usually spreads via local shedding into the peritoneal cavity
followed by implantation on the peritoneum, and via local invasion of bowel and
bladder.  The incidence of positive nodes at primary surgery has been reported
as high as 24$\%$ in patients with stage I disease, 50$\%$ in patient with stage II
disease, 74$\%$ in patients with stage III disease, and 73$\%$ in patients with stage
IV disease.
\cite{6}  In this study, the pelvic nodes were involved as often as the
para-aortic nodes.  Tumor cells may also block diaphragmatic lymphatics.  The
resulting impairment of lymphatic drainage of the peritoneum is thought to play
a role in development of ascites in ovarian cancer.  Also, transdiaphragmatic
spread to the pleura is common.

\setcounter{qC}{0}
Although the ovarian cancer-associated antigen, CA 125, has no prognostic
significance when measured at the time of diagnosis, it has a high correlation
with survival when measured one month after the third course of chemotherapy
for patients with stage III or stage IV disease.
\cite{7}
For patients whose
elevated CA 125 normalizes with chemotherapy, more than one subsequent elevated
CA 125 is highly predictive of active disease, but this does not mandate
immediate therapy.
\cite{8, 9}

% \begin{cbunit}, \end{cbunit} identify start/end of a reference section
% created on a per section basis at the end of each section.
% a \cite{N} refers to the bib entry \bibitem{1} within the 
% \begin{thebibliography}, \end{thebibliography} environment
% The text of the <Citation> tag will be the text of a \bibitem{N} 
% 
\begin{thebibliography}{100}
\bibitem{1} Volker Englisch, The Art of Using \LaTeX . 2001
\bibitem{2} Perry Gonella, Here, there and everywhere.  Proceedings, 1999.
\bibitem{3} Parasuram Iyer, An electron in the cable.  O'Reilley, 1989.
\bibitem{4} Alan Meyer, My life is chess.  Biography, AM Publishing, 1999.
\bibitem{5} Bob Kline, The python and the mouse.  RK Systems's Press, 2001.
\bibitem{6}N Wolmark, H Rockette, DL Wickerham, B Fisher, C Redmond, ER Fisher, 
M Potvin, RJ Davies, J Jones, A Robidoux.  Adjuvant therapy of Dukes' A, B, 
and C adenocarcinoma of the colon with portal-vein fluorouracil hepatic 
infusion: preliminary results of National Surgical Adjuvant Breast and Bowel 
Project Protocol C-02. J Clin Oncol 8(9): 1466-75, 1990.
\bibitem{7}Unresolved footnote for PDQ Citation:4027.
\bibitem{8}M Merson, S Andreola, V Galimberti, R Bufalino, S Marchini, 
U Veronesi.  Breast carcinoma presenting as axillary metastases without 
evidence of a primary tumor. Cancer 70(2): 504-8, 1992.
\bibitem{9}Unresolved footnote for PDQ Citation:11622.
\end{thebibliography}

\end{cbunit}

\begin{cbunit}

\section{Cellular Classification}

\setcounter{qC}{0}
This guy \cite{1} (order entered: 1) is working on the Footnote convertion.
The other guy \cite{2,3} (order entered: 2,3) is working on the CIPS project, too. 
And at the end
we are using all three guys in one citation \cite{2,3,1} (order entered: 2,3,1). 
At the very end
we need a couple couple more guys here \cite{5,3,1,2,4} (order entered: 5,3,1,2,4).

\setcounter{qC}{0}
(Refer to the PDQ summary on Ovarian Low Malignant Potential Tumor Treatment
for more information.)

% \bibliographystyle{alpha}
\begin{thebibliography}{100}
\bibitem{1} Hemant Virkar, The Glass is half full, Research project, 1988.   
\bibitem{2} Kent Hevner , CIPS projects and Government games, 2001. 
\bibitem{3} Spencer Hines, We'll get it done.  Oxford Press. 1999. 
\bibitem{4} Micharl Arluk, The Russian Invasion, WWII Publishing, 1945.
\bibitem{5} Albert Einstein, $E = {m \over c^2}$.  Knowledge Press, 1961.
\end{thebibliography}
\end{cbunit}

\end{document}
