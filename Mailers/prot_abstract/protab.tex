%% LaTeX Sample file for Protocol Abstract Mailers
%%                                   Created: 09/19/2001, Volker Englisch
%% ----------------------------------------------------------------------
%% The XML2LaTeX converter has to create a file with the data structure that
%%is presented here in order to display a properly formatted summary
%%

%% -- START -- Document Declarations and Definitions
\documentclass[12pt]{article}
\usepackage[none,light,bottom]{draftcopy}
%\usepackage[first,light,bottomafter,timestamp]{draftcopy}
\draftcopySetGrey{0.90}
\draftcopyVersion{1.0~~}

% Package and code to set correct double-quotes
% ---------------------------------------------
\usepackage{ifthen}
\newcounter{qC}
\newcommand{\tQ}{%
  \addtocounter{qC}{1}%
  \ifthenelse{\isodd{\value{qC}}}{``}{''}%
}


% Package Fancyhdr for Header/Footer/Reviewer Information
% -------------------------------------------------------
\usepackage{fancyhdr}
\pagestyle{fancy}

% Placing title, date, and board member in header
\fancyhead[C]{{\bfseries Protocol ID:  \ProtocolID} \\}
\fancyhead[L]{\today}
\fancyhead[R]{DocID: \TheDocID}
\fancyfoot[C]{\thepage}
\renewcommand\headrulewidth{1pt}
\renewcommand\footrulewidth{1pt}

% Double-spacing
% \renewcommand\baselinestretch{1.5}
% Wider Margins
% \setlength{\oddsidemargin}{12pt}
% \setlength{\textwidth}{6in}

% Package calc used in following list definition
% ----------------------------------------------
\usepackage{calc}

% Define list environment
% -----------------------
\newcommand{\entrylabel}[1]{\mbox{\bfseries{#1:}}\hfil}
\newenvironment{entry}
   {\begin{list}{}%
       {\renewcommand{\makelabel}{\entrylabel}%
        \setlength{\labelwidth}{170pt}%
       \setlength{\itemsep}{-5pt}%
        \setlength{\leftmargin}{\labelwidth+\labelsep}%
       }%
   }%
{\end{list}}

% Changing Font
% \renewcommand{\familydefault}{phv}

% \ProtocolID
% \TheProtocolID --> XML = <ProtocolIDs.PrimaryID.IDString>
\newcommand{\ProtocolID}{ACRIN-6655}

% \DocID --> XML = ????
\newcommand{\TheDocID}{86539}
%\newcommand{\DocID}[1]{\renewcommand{\TheDocID}{DocID:  #1}}

\setlength{\parskip}{1.2mm}
\setlength{\parindent}{0mm}
\setlength{\headheight}{28pt}

\renewcommand{\thesection}{\hspace{-1.0em}}

%% -- END -- Document Declarations and Definitions

\begin{document}


% Tell fancyhdr package to modify the plain style (plain style is 
% default on a cover page), e.g. put header on first page as well.
% ----------------------------------------------------------------
\fancypagestyle{plain}{%
    % \fancyhf{}%
    \fancyhead[L]{\today}
    \fancyfoot[C]{\thepage}
    \renewcommand\headrulewidth{1pt}
    \renewcommand\footrulewidth{1pt}}

     \makeatletter \renewcommand\@biblabel[1]{#1.} \makeatother
 
\setcounter{qC}{0}
% Protocol Title section --> XML = <TitleText> of 
%                                   <ProtocolTitle.TitleType>=Professional
\subsection*{Protocol Title}
Phase I/III Randomized Study of Irinotecan, Fluorouracil, and Leucovorin
Calcium With or Without Hepatic Chemoembolization in Patients With
Adenocarcinoma of the Colon or Rectum Metastatic to the Liver (Summary
Last Modified 09/2001)

\subsection*{General Protocol Information}
\par
\setcounter{qC}{0}
\begin{entry}
% Internal Protocol Key --> XML = ????
\item[Internal key]               16138 
\item[Protocol ID]                \ProtocolID
% Alternate Protocol IDs --> XML = <ProtocolIDs.OtherID.IDString>
                                  \\ NCI-H92-0023
% Protocol Activation Date --> XML = <CurrentProtocolStatus><Date>
%                                    or
%                                    <PrevProtocolStatus><Date>
%                                    depending on the entry of
%                                    <ProtocolStatusName>
\item[Protocol Activation Date]           
% Lead Organization --> XML = <ProtocolLeadOrg><OrganizationID>
\item[Lead Organization]          American College of Radiology Imaging Network
% Protocol Chairman --> XML = <ProtocolPersonnel><PersonRole>Protocol Chair
%                               <PersonID>
\item[Protocol Chairman]          Michael C. Soulen, M.D.
% Phone --> XML = ????
\item[Phone]  215-662-7111
% Address --> XML = ????
\item[Address] University of Pennsylvania Cancer Center \\
                                  3400 Spruce Street \\
                                  Philadelphia, PA  19104-4283
% Protocol Status --> XML = <CurrentProtocolStatus><ProtocolStatusName>
\item[Protocol Status]            Protocol Status approved

% Eligibility Age Range --> XML = <Eligibility><AgeText>
\item[Eligible Patient Age Range] Any age
% Eligibility Age Range --> XML = <Eligibility><LowAge> 
\item[Lower Age Limit]            0
% Eligibility Age Range --> XML = <Eligibility><HighAge> 
\item[Upper Age Limit]            120
\end{entry}

\setcounter{qC}{0}
% Disease Retrieval Terms --> XML = <Eligibility><Diagnosis>
\subsection*{Disease Retrieval Terms}
\begin{list}{$\circ$}{\setlength{\itemsep}{-5pt}}
\item this is a long one for adenocarcinoma of the colon 
\item adenocarcinoma of the rectum 
\item liver metastases            
\item this is a long one for recurrent colon cancer 
\item recurrent rectal cancer     
\item stage IV colon cancer 
\item stage IV rectal cancer      
\end{list}

\setcounter{qC}{0}
% Protocol Objectives --> XML = <Objectives><Para>
\subsection*{Protocol Objectives}
I.  Determine the survival of patients with liver-dominant metastatic colorectal
adenocarcinoma treated with irinotecan, fluorouracil, and leucovorin calcium
with or without hepatic chemoembolization.
                                     
II.  Determine response in the liver, time to hepatic tumor progression, and
time to extrahepatic tumor progression in these patients treated with these
regimens.
                                     
III.  Determine the possible treatment differences with respect to morbidity,
toxic effects of chemoembolization, toxic effects of chemotherapy, and death
from cancer-related complications in these patients.

\setcounter{qC}{0}
% Patient Eligibility --> XML = <EntryCriteria><Para>
\subsection*{Patient Eligibility}                
--Disease Characteristics--

Histologically confirmed metastatic colorectal adenocarcinoma
Measurable metastasis to liver at least 1.0 cm
No more than 50% liver replacement by tumor
Less than 75% of total liver volume
Known extrahepatic disease limited to lymph nodes and less than 2 cm
No ascites
                                     

Ineligible for surgery
                                     
 


--Prior/Concurrent Therapy--
                                     
Biologic therapy:
No more than 1 prior adjuvant immunotherapy regimen for colon cancer
                                     
Chemotherapy:
At least 6 months since prior adjuvant chemotherapy and recovered
No more than 1 prior adjuvant chemotherapy regimen for colon cancer
No prior hepatic arterial infusion chemotherapy
                                     
Endocrine therapy:
Not specified
                                     
Radiotherapy:
At least 1 month since prior radiotherapy
No prior hepatic radiotherapy
                                     
Surgery:
See Disease Characteristics
At least 1 month since prior surgery
Prior surgical resection or ablation or liver metastases allowed
                                     
Other:
No other concurrent therapy
                                     
--Patient Characteristics--
                                     
Age:
Any age

Performance status:
Zubrod 0-2
                                     
Life expectancy:
Not specified
                                     
Hematopoietic:
Absolute granulocyte count at least 2,000/mm3
Platelet count at least 90,000/mm3
No bleeding diathesis not correctable by standard therapy
                                     
Hepatic:
Bilirubin no greater than 2.0 mg/dL
SGOT no greater than 100 mU/mL
Lactate dehydrogenase no greater than 425 mU/mL
No hepatic encephalopathy
                                     
Renal:
Creatinine no greater than 2.0 mg/dL
                                     
Cardiovascular:
No myocardial infarction within the past 6 months
No evidence of congestive heart failure
No severe peripheral vascular disease that would preclude catheterization
 


No portal vein occlusion without hepatopedal collateral flow demonstrated by
angiography
No portal hypertension with hepatofugal flow
                                     
Other:
No severe allergy or intolerance to contrast media, narcotics, sedatives, or
atropine
No other malignancy within the past 5 years except adequately treated basal
cell or squamous cell skin cancer or carcinoma in situ of the cervix
Not pregnant or nursing
Fertile patients must use effective contraception

% Protocol Outline --> XML = <Outline><Para>
\subsection*{Protocol Outline}
This is a phase I dose-escalation study followed by a phase III randomized,
multicenter study.
                                     
Phase I:
Patients in phase I are sequentially enrolled to 1 of 3 treatment regimens.
                                     
Regimen A:  Patients receive irinotecan IV over 60-90 minutes, leucovorin
calcium IV, and fluorouracil IV over 10 minutes on days 1, 8, 15, and 22.
Patients undergo hepatic embolization with embolic suspension only on day 36.
                                     
Regimen B:  Patients receive chemotherapy as in regimen A.  Patients undergo
hepatic chemoembolization with lower-dose cisplatin, doxorubicin, and mitomycin
on day 36.
                                     
Regimen C:  Patients receive chemotherapy as in regimen A.  Patients undergo
hepatic chemoembolization with higher-dose cisplatin, doxorubicin, and mitomycin
on day 36.
                                     
After 1 week of rest, patients in all regimens receive a second 4-week course of
systemic chemotherapy.
                                     
Cohorts of 3-10 patients are sequentially enrolled until the maximum tolerated
dose (MTD) of chemotherapy and chemoembolization is determined.  The MTD is
defined as the dose preceding that at which at least 4 of 10 patients experience
dose-limiting toxicity.
                                     
Phase III:
Patients are stratified according to liver volume involvement (less than 25% vs
25-50% vs greater than 50% to less than 75%) and participating center.  Patients
are randomized to 1 of 2 treatment arms.
                                     
Arm I:  Patients receive irinotecan IV over 60-90 minutes, leucovorin calcium
IV, and fluorouracil IV over 10 minutes on days 1, 8, 15, and 22.  Treatment
repeats every 6 weeks in the absence of disease progression.
                                     
Arm II:  Patients receive chemotherapy as in arm I.  Patients undergo hepatic
chemoembolization with cisplatin, doxorubicin, and mitomycin on day 36.
Chemotherapy repeats every 6 weeks in the absence of disease progression.
Chemoembolization may repeat every 6 weeks for 2-4 courses as necessary.
                                     
                                     
Patients in phase III are followed every 3 months for 2 years, every 6 months
for 3 years, and then annually thereafter.

\setcounter{qC}{0}
\subsection*{Stratification Parameters}
Not abstracted

\setcounter{qC}{0}
\subsection*{Baseline and Treatment Procedures}
Not abstracted

\setcounter{qC}{0}
\subsection*{Measure of Response}
Not abstracted

\setcounter{qC}{0}
% Projected Accrual --> XML = <Accrual><Para>
\subsection*{Projected Accrual}
A total of 9-18 patients will be accrued for phase I of this study.
Approximately 315 patients will be accrued for phase III of this study within
2.5 years.

\setcounter{qC}{0}
\subsection*{Dosage Schedule}
    Not abstracted

\setcounter{qC}{0}
\subsection*{Dosage Formulation}
    Not abstracted

\setcounter{qC}{0}
% Caveat for use of Disease --> XML = Boiler plate warning.
\subsection*{Caveat for use of Disease}
The purpose of most clinical trials listed in this database is to test new
cancer treatments, or new methods of diagnosing, screening, or preventing
cancer.  Because all potentially harmful side effects are not known before a
trial is conducted, dose and schedule modifications may be required for
participants if they develop side effects from the treatment or test.  The
therapy or test described in this clinical trial is intended for use by clinical
oncologists in carefully structured settings, and may not prove to be more
effective than standard treatment.  A responsible investigator associated with
this clinical trial should be consulted before using this protocol.


% Following Text is Boilerplate

\newpage
Please initial this page and fax or send hard copy to the address below. 

If you are requesting any changes to the submitted document please include 
the edited pages of this document.

You may fax the information to the PDQ Protocol Coordinator at:
\begin{verse}
Fax \#:  301-480-8105
\end{verse}

or send hard copy documents to:
\begin{verse}
PDQ Protocol Coordinator    \\
Attn: CIAT                  \\
Cancer Information Products and Systems, NCI, NIH   \\
6116 Executive Blvd. Suite 3002B MSC-8321           \\
Bethesda, MD 20892-8321
\end{verse}

Please initial here if summary is satisfactory to you. 
\hrulefill

If the study is permanently closed to patient entry, please give approximate 
date of closure. 
\hrulefill

Reason for closure. 
\hrulefill      \newline
\mbox{}\hrulefill \newline
\mbox{}\hrulefill \newline
\mbox{}\hrulefill \newline
\mbox{}\hrulefill

Please list/attach any citations resulting from this study.
\vfill


\end{document}
